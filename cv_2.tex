%%%%%%%%%%%%%%%%%%%%%%%%%%%%%%%%%%%%%%%%%
% Medium Length Graduate Curriculum Vitae
% LaTeX Template
% Version 1.1 (9/12/12)
%
% This template has been downloaded from:
% http://www.LaTeXTemplates.com
%
% Original author:
% Rensselaer Polytechnic Institute (http://www.rpi.edu/dept/arc/training/latex/resumes/)
%
% Important note:
% This template requires the res.cls file to be in the same directory as the
% .tex file. The res.cls file provides the resume style used for structuring the
% document.
%
%%%%%%%%%%%%%%%%%%%%%%%%%%%%%%%%%%%%%%%%%

%----------------------------------------------------------------------------------------
%	PACKAGES AND OTHER DOCUMENT CONFIGURATIONS
%----------------------------------------------------------------------------------------

\documentclass[margin, 10pt]{res} % Use the res.cls style, the font size can be changed to 11pt or 12pt here

\usepackage{helvet} % Default font is the helvetica postscript font
%\usepackage{newcent} % To change the default font to the new century schoolbook postscript font uncomment this line and comment the one above

\setlength{\textwidth}{5.1in} % Text width of the document

\begin{document}

%----------------------------------------------------------------------------------------
%	NAME AND ADDRESS SECTION
%----------------------------------------------------------------------------------------

\moveleft.5\hoffset\centerline{\large\bf Philippe Sasseville} % Your name at the top
 
\moveleft\hoffset\vbox{\hrule width\resumewidth height 1pt}\smallskip % Horizontal line after name; adjust line thickness by changing the '1pt'
 
\moveleft.5\hoffset\centerline{1566 rue Cl\'emenceau} % Your address
\moveleft.5\hoffset\centerline{Montr\'eal, QC, H4H 2R1}
\moveleft.5\hoffset\centerline{(514) 850-9755}
\moveleft.5\hoffset\centerline{philippe.sasseville@polymtl.ca}

%----------------------------------------------------------------------------------------

\begin{resume}

%----------------------------------------------------------------------------------------
%	LANGUES PARL\'Es ET \'ECRITES
%----------------------------------------------------------------------------------------

\section{LANGUES}

{\sl Parl\'e:} Francais, Anglais \\
{\sl \'Ecrite:} Francais, Anglais

%----------------------------------------------------------------------------------------
%	EDUCATION SECTION
%----------------------------------------------------------------------------------------

\section{\'EDUCATION}

{\sl Dipl\^ome d'\'Etudes Coll\'egiales,} Sciences de la nature \\
CEGEP Andr\'e Laurendeau, 2010-2012 

{\sl Baccalaur\'eat,} g\'enie informatique \\
\'Ecole Polytechnique de Montr\'eal, Montr\'eal, QC, 111 cr\'edits sur 120\\
Finissant d\'ecembre 2016
 
%----------------------------------------------------------------------------------------
%	COMPUTER SKILLS SECTION
%----------------------------------------------------------------------------------------

\section{COMP\'ETENCES\\ TECHNIQUES} 

{\sl Langages:} 
C, C++,C\#,Python, Java, x86/x64 Assembleur, VHDL, HTML/CSS, SQL, JavaScript, Objective-C \\
{\sl Logiciels:} 
Visual Studio, Cocoa Touch, Android Studio, Netbeans, Eclipse, Open Office, MSOffice, GNU tool chain \\
{\sl Syt\`emes d'exploitations:} Windows,GNU/Linux,Mac OSX

%----------------------------------------------------------------------------------------
%	STAGES
%----------------------------------------------------------------------------------------
 
\section{ STAGES}

{\sl Stagiaire architecture Positive Train Control (PTC)} \hfill \'Et\'e 2016 \\
Canadian National Railway, Montr\'eal, QC
\begin{itemize} \itemsep -2pt % Reduce space between items
\item Administration du logiciel Enterprise Architect (EA)
\item Production de documentations \'ecrites et vid\'eos sur le fonctionnement de EA
\item Cr\'eation de sch\'emas UML et cas d'utilisations reli\'es au programme PTC
\end{itemize}

{\sl D\'eveloppeur stagiaire pour une application bancaire} \hfill \'Et\'e 2015 \\
Soci\'et\' e G\'enerale, Centre de solutions MSC, \'Equipe EFTS (Equity Finance Trading System), Montr\'eal, QC
\begin{itemize} \itemsep -2pt % Reduce space between items
\item Profilage d'une application J2EE et \'elimination d'une fuite m\'emoire
\item  D\'eveloppement de divers scripts (D\'eploiement, v\'erification, data dump)
\item  Installation et gestion d'un syst\`eme de collecte des registres (logs) avec ELK
\end{itemize}
 
%----------------------------------------------------------------------------------------
%	PROFESSIONAL WORK EXPERIENCE SECTION
%----------------------------------------------------------------------------------------
 
\section{\'EXPERIENCES \\ DE TRAVAIL}

{\sl R\'ep\'etiteur de Laboratoire} \hfill Automne 2014 \\
Cours de programmation proc\'edurale (INF1005C)
\begin{itemize} \itemsep -2pt % Reduce space between items
\item R\'epondre aux questions des \'etudiants durant les p\'eriodes de laboratoire
\end{itemize}

%----------------------------------------------------------------------------------------
%	PROJETS
%----------------------------------------------------------------------------------------
 
\section{PROJETS}

{\sl Exploration architechturale et prototypage rapide sur FPGA : codec h264}
\begin{itemize} \itemsep -2pt % Reduce space between items
\item \'Etudes des d\'ependances et du parall\'elisme avec Pareon
\item Modularisation et exploration architechturale
\item Optimisation des modules en mat\'eriel avec vivado HLS
\end{itemize} 

{\sl Syst\`eme de gestion pour restaurant embarqu\'e sur FPGA avec interface sur tablette}
\begin{itemize} \itemsep -2pt % Reduce space between items
\item D\'eveloppement de l'application Android avec android studio
\item  Impl\'ementation de communication par messages HTTP1.1 avec API REST
\item  Programmation du back end en C sur processeurs ARM
\end{itemize} 

%----------------------------------------------------------------------------------------
%	PRIX
%----------------------------------------------------------------------------------------
 
\section{PRIX}

{\sl Laur\'eat de la question Deloitte, comp\'etition informatique CEGL/CEGInfo 2016}

\end{resume}
\end{document}